% Options for packages loaded elsewhere
\PassOptionsToPackage{unicode}{hyperref}
\PassOptionsToPackage{hyphens}{url}
%
\documentclass[
]{article}
\usepackage{lmodern}
\usepackage{amsmath}
\usepackage{ifxetex,ifluatex}
\ifnum 0\ifxetex 1\fi\ifluatex 1\fi=0 % if pdftex
  \usepackage[T1]{fontenc}
  \usepackage[utf8]{inputenc}
  \usepackage{textcomp} % provide euro and other symbols
  \usepackage{amssymb}
\else % if luatex or xetex
  \usepackage{unicode-math}
  \defaultfontfeatures{Scale=MatchLowercase}
  \defaultfontfeatures[\rmfamily]{Ligatures=TeX,Scale=1}
\fi
% Use upquote if available, for straight quotes in verbatim environments
\IfFileExists{upquote.sty}{\usepackage{upquote}}{}
\IfFileExists{microtype.sty}{% use microtype if available
  \usepackage[]{microtype}
  \UseMicrotypeSet[protrusion]{basicmath} % disable protrusion for tt fonts
}{}
\makeatletter
\@ifundefined{KOMAClassName}{% if non-KOMA class
  \IfFileExists{parskip.sty}{%
    \usepackage{parskip}
  }{% else
    \setlength{\parindent}{0pt}
    \setlength{\parskip}{6pt plus 2pt minus 1pt}}
}{% if KOMA class
  \KOMAoptions{parskip=half}}
\makeatother
\usepackage{xcolor}
\IfFileExists{xurl.sty}{\usepackage{xurl}}{} % add URL line breaks if available
\IfFileExists{bookmark.sty}{\usepackage{bookmark}}{\usepackage{hyperref}}
\hypersetup{
  pdftitle={R Notebook},
  hidelinks,
  pdfcreator={LaTeX via pandoc}}
\urlstyle{same} % disable monospaced font for URLs
\usepackage[margin=1in]{geometry}
\usepackage{color}
\usepackage{fancyvrb}
\newcommand{\VerbBar}{|}
\newcommand{\VERB}{\Verb[commandchars=\\\{\}]}
\DefineVerbatimEnvironment{Highlighting}{Verbatim}{commandchars=\\\{\}}
% Add ',fontsize=\small' for more characters per line
\usepackage{framed}
\definecolor{shadecolor}{RGB}{248,248,248}
\newenvironment{Shaded}{\begin{snugshade}}{\end{snugshade}}
\newcommand{\AlertTok}[1]{\textcolor[rgb]{0.94,0.16,0.16}{#1}}
\newcommand{\AnnotationTok}[1]{\textcolor[rgb]{0.56,0.35,0.01}{\textbf{\textit{#1}}}}
\newcommand{\AttributeTok}[1]{\textcolor[rgb]{0.77,0.63,0.00}{#1}}
\newcommand{\BaseNTok}[1]{\textcolor[rgb]{0.00,0.00,0.81}{#1}}
\newcommand{\BuiltInTok}[1]{#1}
\newcommand{\CharTok}[1]{\textcolor[rgb]{0.31,0.60,0.02}{#1}}
\newcommand{\CommentTok}[1]{\textcolor[rgb]{0.56,0.35,0.01}{\textit{#1}}}
\newcommand{\CommentVarTok}[1]{\textcolor[rgb]{0.56,0.35,0.01}{\textbf{\textit{#1}}}}
\newcommand{\ConstantTok}[1]{\textcolor[rgb]{0.00,0.00,0.00}{#1}}
\newcommand{\ControlFlowTok}[1]{\textcolor[rgb]{0.13,0.29,0.53}{\textbf{#1}}}
\newcommand{\DataTypeTok}[1]{\textcolor[rgb]{0.13,0.29,0.53}{#1}}
\newcommand{\DecValTok}[1]{\textcolor[rgb]{0.00,0.00,0.81}{#1}}
\newcommand{\DocumentationTok}[1]{\textcolor[rgb]{0.56,0.35,0.01}{\textbf{\textit{#1}}}}
\newcommand{\ErrorTok}[1]{\textcolor[rgb]{0.64,0.00,0.00}{\textbf{#1}}}
\newcommand{\ExtensionTok}[1]{#1}
\newcommand{\FloatTok}[1]{\textcolor[rgb]{0.00,0.00,0.81}{#1}}
\newcommand{\FunctionTok}[1]{\textcolor[rgb]{0.00,0.00,0.00}{#1}}
\newcommand{\ImportTok}[1]{#1}
\newcommand{\InformationTok}[1]{\textcolor[rgb]{0.56,0.35,0.01}{\textbf{\textit{#1}}}}
\newcommand{\KeywordTok}[1]{\textcolor[rgb]{0.13,0.29,0.53}{\textbf{#1}}}
\newcommand{\NormalTok}[1]{#1}
\newcommand{\OperatorTok}[1]{\textcolor[rgb]{0.81,0.36,0.00}{\textbf{#1}}}
\newcommand{\OtherTok}[1]{\textcolor[rgb]{0.56,0.35,0.01}{#1}}
\newcommand{\PreprocessorTok}[1]{\textcolor[rgb]{0.56,0.35,0.01}{\textit{#1}}}
\newcommand{\RegionMarkerTok}[1]{#1}
\newcommand{\SpecialCharTok}[1]{\textcolor[rgb]{0.00,0.00,0.00}{#1}}
\newcommand{\SpecialStringTok}[1]{\textcolor[rgb]{0.31,0.60,0.02}{#1}}
\newcommand{\StringTok}[1]{\textcolor[rgb]{0.31,0.60,0.02}{#1}}
\newcommand{\VariableTok}[1]{\textcolor[rgb]{0.00,0.00,0.00}{#1}}
\newcommand{\VerbatimStringTok}[1]{\textcolor[rgb]{0.31,0.60,0.02}{#1}}
\newcommand{\WarningTok}[1]{\textcolor[rgb]{0.56,0.35,0.01}{\textbf{\textit{#1}}}}
\usepackage{graphicx}
\makeatletter
\def\maxwidth{\ifdim\Gin@nat@width>\linewidth\linewidth\else\Gin@nat@width\fi}
\def\maxheight{\ifdim\Gin@nat@height>\textheight\textheight\else\Gin@nat@height\fi}
\makeatother
% Scale images if necessary, so that they will not overflow the page
% margins by default, and it is still possible to overwrite the defaults
% using explicit options in \includegraphics[width, height, ...]{}
\setkeys{Gin}{width=\maxwidth,height=\maxheight,keepaspectratio}
% Set default figure placement to htbp
\makeatletter
\def\fps@figure{htbp}
\makeatother
\setlength{\emergencystretch}{3em} % prevent overfull lines
\providecommand{\tightlist}{%
  \setlength{\itemsep}{0pt}\setlength{\parskip}{0pt}}
\setcounter{secnumdepth}{-\maxdimen} % remove section numbering
\ifluatex
  \usepackage{selnolig}  % disable illegal ligatures
\fi

\title{R Notebook}
\author{}
\date{\vspace{-2.5em}}

\begin{document}
\maketitle

\begin{Shaded}
\begin{Highlighting}[]
\CommentTok{\#install.packages("philentropy")}
\CommentTok{\#install.packages("readxl")}
\end{Highlighting}
\end{Shaded}

\begin{Shaded}
\begin{Highlighting}[]
\FunctionTok{library}\NormalTok{(philentropy)}
\end{Highlighting}
\end{Shaded}

\begin{verbatim}
## Warning: package 'philentropy' was built under R version 4.0.5
\end{verbatim}

\begin{Shaded}
\begin{Highlighting}[]
\FunctionTok{library}\NormalTok{(readxl)}
\end{Highlighting}
\end{Shaded}

\begin{Shaded}
\begin{Highlighting}[]
\DocumentationTok{\#\# Import data}
\NormalTok{data }\OtherTok{\textless{}{-}} \FunctionTok{read\_excel}\NormalTok{(}\StringTok{"DataCoba.xlsx"}\NormalTok{)}
\NormalTok{data }\OtherTok{\textless{}{-}}\NormalTok{ data[}\DecValTok{1}\SpecialCharTok{:}\DecValTok{8}\NormalTok{, }\DecValTok{2}\SpecialCharTok{:}\DecValTok{4}\NormalTok{]}

\NormalTok{data }\OtherTok{\textless{}{-}} \FunctionTok{as.matrix}\NormalTok{(data)}
\NormalTok{data}
\end{Highlighting}
\end{Shaded}

\begin{verbatim}
##      Feat 1 Feat 2 Feat 3
## [1,]     15      0      5
## [2,]      0     20      0
## [3,]      6      3      5
## [4,]      2      7      0
## [5,]      1      0      3
## [6,]      0      1      1
## [7,]      8      9      3
## [8,]      5      1      7
\end{verbatim}

\hypertarget{l1-norm}{%
\subsection{L1 norm}\label{l1-norm}}

L1 Norm atau juga biasa dikenal sebagai \textbf{Manhattan Distance} atau
\textbf{Taxicab Norm} adalah penjumlahan dari besaran vektor pada suatu
ruang.

Berikut contoh manhattan distance dari data point dengan nilai (0,0)
dengan (3,4)

\includegraphics{images/1.jpg}

Jarak Manhattan distance dari kedua data point tersebut adalah

\includegraphics{images/L1Example.png}

Manhattan distance ini adalah cara paling umum untuk mengukur jarak
antar vektor, yaitu jumlah dari selisih absolut dari tiap-tiap komponen
vektor.

\includegraphics{images/L1Formula.png}

\begin{Shaded}
\begin{Highlighting}[]
\NormalTok{L1 }\OtherTok{\textless{}{-}} \FunctionTok{distance}\NormalTok{(data, }\AttributeTok{method=}\StringTok{"manhattan"}\NormalTok{)}
\NormalTok{df\_L1 }\OtherTok{=} \FunctionTok{as.data.frame}\NormalTok{(}\FunctionTok{t}\NormalTok{(L1)) }\CommentTok{\# Change to dataframe}
\NormalTok{df\_L1}
\end{Highlighting}
\end{Shaded}

\begin{verbatim}
##    v1 v2 v3 v4 v5 v6 v7 v8
## v1  0 40 12 25 16 20 18 13
## v2 40  0 28 15 24 20 22 31
## v3 12 28  0 13 10 12 10  5
## v4 25 15 13  0 11  9 11 16
## v5 16 24 10 11  0  4 16  9
## v6 20 20 12  9  4  0 18 11
## v7 18 22 10 11 16 18  0 15
## v8 13 31  5 16  9 11 15  0
\end{verbatim}

\begin{Shaded}
\begin{Highlighting}[]
\CommentTok{\# Export to csv}
\FunctionTok{write.csv}\NormalTok{(df\_L1, }\StringTok{"export/L1.csv"}\NormalTok{)}
\end{Highlighting}
\end{Shaded}

\hypertarget{l2-norm}{%
\subsection{L2 norm}\label{l2-norm}}

L2 norm atau yang biasa dikenal dengan \textbf{Euclidean distance}
adalah jarak terdekat pada antar \textbf{data point}.

Berikut contoh euclidean distance dari data point dengan nilai (0,0)
dengan (3,4)

\begin{figure}
\centering
\includegraphics{images/2.jpg}
\caption{Contoh gambar penjelasan euclidean distance}
\end{figure}

Jarak Euclidean dari kedua data point tersebut adalah

\includegraphics{images/L2Example.png}

Euclidean distance ini adalah cara yang paling sering digunakan untuk
mengukur jarak antar dua titik.

\includegraphics{images/L2Formula.png}

\begin{Shaded}
\begin{Highlighting}[]
\NormalTok{L2 }\OtherTok{\textless{}{-}} \FunctionTok{distance}\NormalTok{(data, }\AttributeTok{method=}\StringTok{"euclidean"}\NormalTok{)}
\NormalTok{df\_L2 }\OtherTok{=} \FunctionTok{as.data.frame}\NormalTok{(}\FunctionTok{t}\NormalTok{(L2)) }\CommentTok{\# Change to dataframe}
\NormalTok{df\_L2}
\end{Highlighting}
\end{Shaded}

\begin{verbatim}
##           v1       v2        v3        v4        v5        v6        v7
## v1  0.000000 25.49510  9.486833 15.588457 14.142136 15.556349 11.575837
## v2 25.495098  0.00000 18.708287 13.152946 20.248457 19.026298 13.928388
## v3  9.486833 18.70829  0.000000  7.549834  6.164414  7.483315  6.633250
## v4 15.588457 13.15295  7.549834  0.000000  7.681146  6.403124  7.000000
## v5 14.142136 20.24846  6.164414  7.681146  0.000000  2.449490 11.401754
## v6 15.556349 19.02630  7.483315  6.403124  2.449490  0.000000 11.489125
## v7 11.575837 13.92839  6.633250  7.000000 11.401754 11.489125  0.000000
## v8 10.246951 20.85665  3.000000  9.695360  5.744563  7.810250  9.433981
##           v8
## v1 10.246951
## v2 20.856654
## v3  3.000000
## v4  9.695360
## v5  5.744563
## v6  7.810250
## v7  9.433981
## v8  0.000000
\end{verbatim}

\begin{Shaded}
\begin{Highlighting}[]
\CommentTok{\# Export to csv}
\FunctionTok{write.csv}\NormalTok{(df\_L2, }\StringTok{"export/L2.csv"}\NormalTok{)}
\end{Highlighting}
\end{Shaded}

\hypertarget{l}{%
\subsection{L∞}\label{l}}

L infinite atau \textbf{L supremum} adalah jarak nilai maksimum secara
absolut antara tiap elemen vektor.

Berikut contoh L Supremum :

\includegraphics{images/LSup Formula.png}

\includegraphics{images/LSup Answer.png}

Maka didapatkan lah L Sup dari Vektor A dengan C yaitu bernilai 20

\begin{Shaded}
\begin{Highlighting}[]
\NormalTok{LSup }\OtherTok{\textless{}{-}} \FunctionTok{distance}\NormalTok{(data, }\AttributeTok{method=}\StringTok{"chebyshev"}\NormalTok{)}
\NormalTok{df\_LSup }\OtherTok{=} \FunctionTok{as.data.frame}\NormalTok{(}\FunctionTok{t}\NormalTok{(LSup)) }\CommentTok{\# Change to dataframe}
\NormalTok{df\_LSup}
\end{Highlighting}
\end{Shaded}

\begin{verbatim}
##    v1 v2 v3 v4 v5 v6 v7 v8
## v1  0 20  9 13 14 15  9 10
## v2 20  0 17 13 20 19 11 19
## v3  9 17  0  5  5  6  6  2
## v4 13 13  5  0  7  6  6  7
## v5 14 20  5  7  0  2  9  4
## v6 15 19  6  6  2  0  8  6
## v7  9 11  6  6  9  8  0  8
## v8 10 19  2  7  4  6  8  0
\end{verbatim}

\begin{Shaded}
\begin{Highlighting}[]
\CommentTok{\# Export to csv}
\FunctionTok{write.csv}\NormalTok{(df\_LSup, }\StringTok{"export/LSup.csv"}\NormalTok{)}
\end{Highlighting}
\end{Shaded}

\hypertarget{cosine}{%
\section{Cosine}\label{cosine}}

Cosine Similarity adalah ukuran kemiripan antara 2 vektor. NIlai cosine
similarity berada pada range antara 0 dan 1.

Berikut adalah rumus hitung dari Cosine Similarity

\includegraphics{images/CosineFormula.png}

Analogi dari cosine similarity :

\includegraphics{images/Cosine.png}

Contoh Cosine :

\includegraphics{images/Cosine2.png}

Penyelesaian :

\includegraphics{images/Cosine3.png}

\begin{Shaded}
\begin{Highlighting}[]
\CommentTok{\# Cosine}
\NormalTok{LCosine }\OtherTok{\textless{}{-}} \FunctionTok{distance}\NormalTok{(data, }\AttributeTok{method=}\StringTok{"cosine"}\NormalTok{)}
\NormalTok{df\_LCosine }\OtherTok{=} \FunctionTok{as.data.frame}\NormalTok{(}\FunctionTok{t}\NormalTok{(LCosine))}
\NormalTok{df\_LCosine}
\end{Highlighting}
\end{Shaded}

\begin{verbatim}
##           v1        v2        v3         v4         v5        v6        v7
## v1 1.0000000 0.0000000 0.8693183 0.26062335 0.60000000 0.2236068 0.6880237
## v2 0.0000000 1.0000000 0.3585686 0.96152395 0.00000000 0.7071068 0.7252407
## v3 0.8693183 0.3585686 1.0000000 0.54178501 0.79372539 0.6761234 0.8668284
## v4 0.2606233 0.9615239 0.5417850 1.00000000 0.08687445 0.6799001 0.8744375
## v5 0.6000000 0.0000000 0.7937254 0.08687445 1.00000000 0.6708204 0.4332001
## v6 0.2236068 0.7071068 0.6761234 0.67990010 0.67082039 1.0000000 0.6837635
## v7 0.6880237 0.7252407 0.8668284 0.87443754 0.43320011 0.6837635 1.0000000
## v8 0.8033264 0.1154701 0.9384892 0.26963754 0.94938577 0.6531973 0.6513389
##           v8
## v1 0.8033264
## v2 0.1154701
## v3 0.9384892
## v4 0.2696375
## v5 0.9493858
## v6 0.6531973
## v7 0.6513389
## v8 1.0000000
\end{verbatim}

\begin{Shaded}
\begin{Highlighting}[]
\CommentTok{\# Export to csv}
\FunctionTok{write.csv}\NormalTok{(df\_LCosine, }\StringTok{"export/LCosine.csv"}\NormalTok{)}
\end{Highlighting}
\end{Shaded}


\end{document}
